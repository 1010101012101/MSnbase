\paragraph{Speed and memory requirements} Raw mass spectrometry file are generally several 
hundreds of MB large and most of this is used for binary raw spectrum data. As such, 
data containers can easily grow very large and thus require large amounts of RAM. 
This requirement is being tackled by avoiding to load the raw data into memory
and using on-disk random access to the content of \texttt{mzXML}/\texttt{mzML} data files on demand. 
When focusing on reporter ion quantitation, a direct solution for this is to trim the 
spectra using the \Robject{trimMz} method to select the area of interest and thus 
substantially reduce the size of the \Robject{Spectrum} objects. This is illustrated in 
section \ref{sec:trim} on page \pageref{trimMz-example} of the \texttt{MSnbase-demo} vignette.

The independent handling of spectra is ideally suited for parallel processing. 
The \Rfunction{quantify} method now performs reporter peaks quantitation in parallel. 
More functions are being updated.




